\section{A Cloud for High Performance Computing}
With support for GPU Passthrough, SR-IOV, and GPUDirect, we have the building
blocks for a high performance, heterogeneous cloud.  In addition, other common
accelerators (e.g. Xeon Phi~\cite{Phi}) have similarly been demonstrated in
virtualized environments.  Our vision is of a heterogeneous cloud, supporting
both high speed networking and accelerators for tightly coupled applications.

To this end we have developed a heterogeneous cloud based on
OpenStack~\cite{www-openstack}.  In our previous work, we
have demonstrated the ability to rapidly provision GPU, bare metal, and other
heterogeneous resources within a single cloud~\cite{crago2011heterogeneous}.
Building on this effort we have added support for GPU passthrough to OpenStack
as well as SR-IOV support for both ConnectX-2 and ConnectX-3 Infiniband devices.
Mellanox separately supports an OpenStack InfiniBand networking plugin for
OpenStack's Neutron service~\cite{ML2}, however the Mellanox plugin depends on
the ConnectX-3 adapter.  Our institutional requirements depend on ConnecteX-2
SR-IOV support, requiring an independent implementation.

OpenStack supports services for networking (Neutron), compute (Nova), identity
(Keystone), storage (Cinder, Swift), and others.  Our work focuses entirely
on the compute service.  

Scheduling is implemented at two levels: the cloud-level and the node-level.  In
our earlier work, we have developed a cloud-level heterogeneous scheduler for OpenStack,
allowing scheduling based on architectures and
resources~\cite{crago2011heterogeneous}.  In this model, the cloud-level
scheduler dispatches jobs to nodes based on resource requirements (e.g. Kepler
GPU) and node-level resource availability.

At the node, a second level of scheduling occurs to ensure that resources are
tracked and not overcommitted.  Unlike traditional cloud paradigms, devices
passed into VMs cannot be overcommitted.  We treat devices, whether GPUs or
InfiniBand virtual functions, as schedulable resources.  Thus, it is the responsibility of the
individual node to track resources committed and report availability to the
cloud-level scheduler.  For reporting, we piggyback on top of OpenStack's
existing reporting mechanism to provide a low overhead solution.


